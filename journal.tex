% Please do not change the document class
\documentclass{scrartcl}

% Please do not change these packages
\usepackage[hidelinks]{hyperref}
\usepackage[none]{hyphenat}
\usepackage{setspace}
\doublespace

% You may add additional packages here
\usepackage{amsmath}

% Please include a clear, concise, and descriptive title
\title{Research Journal}

% Please do not change the subtitle
\subtitle{COMP210- Research Journal}

% Please put your student number in the author field
\author{1607804}

\begin{document}

\maketitle

\section{Hardware Interfaces for VR Applications: Evaluation on Prototypes\cite{mentzelopoulos2015hardware}}

In this paper, the author discusses an evaluation of two games "Erebus" and "Dreams".
The two games are both virtual reality games (VR); they are both developed in Unreal Engine.
From my understanding, in both games, the users wear a head-mounted display. However, one game uses an Xbox controller while for the other, the participants use a Hydra controller. 

A few things make me question the validity of the results of the usability test.
Firstly, the participants of the test were all studying either a computing related course or a game development course. 
The authors of the paper state that the all the participants have prior game experience, but no experience with controllers while wearing an HMD(head mounted display). 
Arguably this may make it easier for them to understand the controls and how to play the games.

They conclude by suggesting that the participants preferred the Hydra controller with the gesture recognition rather than the standard Xbox controller. 
Also, motion sickness came up as a big problem for both games. The motion sickness caused by VR headsets is still a significant problem within the industry\cite{fernandes2016combating, hettinger1992visually, von2016cyber}. In the paper, "Real virtuality: a code of ethical conduct. Recommendations for good scientific practice and the consumers of VR-technology" \cite{madary2016real} they mention that the risk of motion sickness must be minimised\cite{behr2005some}. After the first few participants, was it ethical for them to continue the study knowing that it caused motion sickness? The authors do not mention warning the users about the potential risk.  Not informing the users may be seen as a violation of ethics, as it could harm those taking part in the study. 

\section{Exploitation of heuristics for virtual environments \cite {hvannberg2012exploitation}}
This paper, in summary, is a study on the overuse of the heuristics methods suggested by Sutcliffe and Gault in their article "Heuristic evaluation of virtual reality applications"\cite{sutcliffe2004heuristic}. In particular, it looks at the papers that reference Sutcliffe and Gaulf and how they use their methodology. Using this knowledge, the authors discovered that only one in five citations used this method either entirely or partly.

The authors state that "not a single heuristics list may be useful to help evaluators uncover problems"\cite [p.312] {hvannberg2012exploitation}. A single heuristics list cannot be applied to different virtual environments and expect to be as effective in each. Instead, they suggest a patchwork where there are specialised lists of heuristics that are dependent on the sector of the virtual environment.

\section{Investigating the Balance between Virtuality and Reality in Mobile Mixed Reality UI Design – User Perception of an Augmented City\cite{venta2014investigating}}
The authors of this paper conducted a study to investigate the perception that users will have to the UI of a mobile mixed reality app. More specifically they compared two UI solutions one of which used a camera-based AR(augmented reality\cite{azuma2001recent, milgram1994taxonomy}) view, the other being a 3D model of the city. The online questionnaire and the field results favouring the camera based display over the 3D model based display. The results showed that the most appealing aspect of the 3D model of the city was the lack of moving elements (people, shadows, etc.). However, this may also be why it is not as popular with the users. Some papers suggest that its a requirement of AR, AV (augmented virtuality\cite{jang2011overlapping}) to have a real-time reflection of the real world into the virtual world\cite {milgram1994taxonomy,jang2011overlapping}. Perhaps the lack of the shadows and people were the cause of the users disconnect from the 3D world.

I question the validity of the online survey when it comes to the user interface reviews. The online questionnaire participants only saw screenshots of the UI and were asked their opinions and preferences. At no point were they able to interact with it as they should when performing a usability test\cite{rubin2008handbook,nielsen1994usability}. The study of this paper focuses on the perception; the argument still stands, the online users' perception may have changed if they used the software.

\section {Homuncular flexibility in virtual reality\cite{won2015homuncular}}
The two experiments in this paper aim to document what happens when you connect a human to a virtual avatar that moves in novel ways. To do this, they use a virtual reality headset to simulate the changes in their body. The second experiment had some of the participants using three arms the extra arm is long enough to hit the targets further away without needing to step. The other participants had just there normal two arms, to reach the second set of targets they had to step. The results showed that participants using an avatar with three arms could hit more targets than ones in a body with two arms.

 One thing not mentioned in this that \cite{mentzelopoulos2015hardware} discuss is that of motion sickness. Perhaps controlling a different shaped body would induce more motion sickness due to the disorientation of the body transformation. A similar study performed by another group found similar results\cite{steptoe2013human}. The participants responded to threats to their new body parts (tails) and quickly learned how to remap normal degrees so that they could control their virtual forms. Further research should be done to see if one can build muscle memory with these appendages. For example when under threat using their third arm as a protection mechanism\cite{chafe1996musical}.

\section {(Re-) examination of multimodal augmented reality\cite{rosa2016re}}
This paper explores older works of literature but with a focus on three kinds of stimuli which the author defines using work from \cite{bach2003sensory,milgram1994taxonomy,bayart2006haptic}. Firstly, the author defines Real as a stimulus that originated from the physical environment and experienced in its natural form. Mediated is where the stimulus is from the physical environment and manipulated in some way. Lastly, they define virtual as a stimulus that is computer-generated and one experiences through hardware designed for that purpose. 

Examples of Mediated-Virtual can be seen in the \cite{venta2014investigating} with their 3D model of the city. The city buildings are virtual, but they derive from the real world buildings.  Their camera-based version is Real-Virtual as its seen through the phone and interacted with on the phone but is an unedited version of the physical structure. 

\section {Real virtuality: a code of ethical conduct. recommendations for good scientific practice and the consumers of vr-technology\cite{madary2016real}}
This paper should be referred to by all who want to research or use VR products. It specifies good practice when it comes to research in VR giving recommendations and suggesting in which areas greater ethical deliberation will be needed. The authors also go on to describe the potential risks to society and civilians that VR will represent. Interjected throughout this paper are philosophical concepts such as "conscious experience exactly is a virtual model of the world".

As previously discussed in this journal the authors of \cite{madary2016real} identify motion sickness as a factor that one must minimise in their "Research Ethics of VR" section. Another issue \cite{madary2016real} raise is that of informed consent of the psychological effects of VR.  To back up their claims they use the virtual pit experiment\cite{meehan2002physiological}. Works from \cite{rosenberg2013virtual,hershfield2011increasing,rizzo1998basic} support the potential for lasting effect of VR. 
\bibliographystyle{ieeetran}
\bibliography{references}

\end{document}
